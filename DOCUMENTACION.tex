\documentclass[12pt]{article}
\usepackage[spanish]{babel}
\usepackage[utf8]{inputenc}
\usepackage{geometry}
\geometry{a4paper, margin=2.5cm}
\usepackage{graphicx}
\usepackage{hyperref}
\title{\textbf{Documentación Proyecto E-commerce tipo Mercado Libre}}
\author{andresva9@hotmail.com}
\date{Julio 2025}
\begin{document}
\maketitle
\tableofcontents
\newpage

\section{Introducción}
Este documento describe las decisiones de diseño, tecnologías utilizadas, arquitectura, desafíos y ejemplos de uso del proyecto e-commerce tipo Mercado Libre.

\section{Elección de Diseño}
\begin{itemize}
  \item \textbf{Arquitectura:} Separación en frontend (React + Vite) y backend (Node.js + Express).
  \item \textbf{Estilos:} Inspirados en Mercado Libre, con CSS modular y responsivo.
  \item \textbf{Testing:} Jest, React Testing Library y Supertest para asegurar calidad.
  \item \textbf{Versionamiento:} Git + GitHub con autenticación SSH.
\end{itemize}

\section{Tecnologías Utilizadas}
\begin{itemize}
  \item \textbf{React:} Interfaces de usuario basadas en componentes.
  \item \textbf{Vite:} Herramienta de desarrollo rápida para frontend.
  \item \textbf{Node.js y Express:} Backend eficiente y escalable.
  \item \textbf{Jest, React Testing Library, Supertest:} Pruebas automáticas.
  \item \textbf{Babel:} Soporte para JSX y ESM.
  \item \textbf{CSS:} Personalización visual y responsividad.
  \item \textbf{Git + GitHub:} Control de versiones y colaboración.
\end{itemize}

\section{Arquitectura y Estructura del Proyecto}
\subsection*{Estructura General}
\begin{itemize}
  \item \textbf{frontend/} (React + Vite): src, public, test, estilos.
  \item \textbf{backend/} (Node.js + Express): index.js, product.json, test.
  \item \textbf{Configuración:} package.json, babel.config.mjs, jest.config.cjs.
\end{itemize}

\subsection*{Componentes Principales}
\begin{itemize}
  \item \textbf{App.jsx:} Estructura principal y rutas.
  \item \textbf{Navbar.jsx:} Navegación superior.
  \item \textbf{ProductList.jsx:} Listado y filtros de productos.
  \item \textbf{ProductDetail.jsx:} Detalle de producto.
  \item \textbf{Cart.jsx:} Carrito de compras.
  \item \textbf{CartContext.jsx:} Estado global del carrito.
  \item \textbf{Analytics.jsx:} Métricas y analítica.
\end{itemize}

\section{Diagrama de Arquitectura}
\begin{center}
\includegraphics[width=0.8\textwidth]{arquitectura.png}
\end{center}
\textit{Nota: El diagrama puede ser editado en draw.io usando el archivo arquitectura.drawio.xml}

\section{Desafíos y Soluciones}
\begin{itemize}
  \item \textbf{Testing y compatibilidad:} Ajuste de Babel y Jest para JSX/ESM.
  \item \textbf{Estilos:} CSS modular y pruebas visuales.
  \item \textbf{Estado global:} Uso de React Context.
  \item \textbf{Versionamiento:} Migración a SSH.
  \item \textbf{Documentación:} README y diagramas claros.
\end{itemize}

\section{Ejemplo de Uso}
\subsection*{Flujo Básico}
\begin{enumerate}
  \item Usuario visualiza productos y filtra.
  \item Accede al detalle y agrega al carrito.
  \item Gestiona el carrito y finaliza compra.
  \item Consulta métricas en Analytics.
\end{enumerate}

\subsection*{Ejemplo de Código}
\begin{verbatim}
import { useContext } from 'react';
import { CartContext } from './CartContext';

function AddToCartButton({ product }) {
  const { addToCart } = useContext(CartContext);
  return (
    <button onClick={() => addToCart(product)}>
      Agregar al carrito
    </button>
  );
}
\end{verbatim}

\section{Conclusión}
El proyecto implementa una arquitectura moderna, modular y escalable, con buenas prácticas de desarrollo, pruebas y documentación, facilitando su mantenimiento y evolución.

\end{document}
